\ihead{Reflexion der erzielten Ergebnisse}
\section{Reflexion der erzielten Ergebnisse}
\label{sec:reflErg}

Das Projektziel, einen funktionsfähigen Prototypen aus abgeänderten und selbst erstellten Komponenten zu entwickeln, ist als erreicht anzusehen. Die gewünschte Funktionalität ist implementiert, wenn auch, aufgrund technischer Hindernisse in der Produktivumgebung, noch nicht getestet. Das Debugging der letzten Änderungen steht also noch an. Ist dieser Prozess erledigt, ist das System in der Lage, Projekte und Aufgaben nach Fähigkeiten zu planen, mit dem bestehenden Verrechnungssystem zu interagieren, einen Überblick über die eigene geplante Auslastung in Zukunft zu erlangen und sich selbst Aufgaben, welche dem eigenen Skillset entsprechen, zuzuordnen.

Um die Debug- und Test-Tätigkeiten zu erleichtern, habe ich neben der Benutzer-, Entwickler- und Administrationsanleitung des Projekts, in welcher ich versucht habe, alle Bestandteile meines Systems genau zu erläutern, ein ausführliches ToDo-Dokument erstellt und die neuen Projektverantwortlichen im Rahmen eines Gesprächs eingewiesen. Diese Maßnahmen sollten den Handover des Projekts also erleichtern und den mittelfristigen Fortschritt des Projekts beschleunigen.
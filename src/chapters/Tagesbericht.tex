\ihead{Tagesberichte}
\section{Tagesberichte}
\label{sec:tagesbericht}

\begin{itshape}
Information für den Lesenden: Die Angaben zu Beginn und Ende der Arbeitstage schließen eine Mittagspause von durchschnittlich 45min ein.
\end{itshape}

\subsection{Februar}
\subsubsection*{1. Februar - 9:00 bis 16:00 Uhr}
Der erste Arbeitstag begann damit, dass ich freundlich von meinem Betreuer, dem operativen Geschäftsführer Dipl.-Ing. Stefan Wambacher, in Empfang genommen wurde. Nachdem ich mich allen Kollegen, die an diesem Tag anwesend waren, vorgestellt habe, wurde ich von meinem Betreuer in die Grundzüge der Betriebsinternen Problemstellung, welche ich in den drei Monaten des Praktikums lösen soll, eingeführt.

Nachdem sich mein Arbeitsnotebook noch in der Postzustellung befunden hatte, war ich am ersten Tag in meiner Tätigkeit insoweit eingeschränkt, als ich keinen Zugang zu den firmeninternen Systemen hatte. Jedoch brachte ich an meinem ersten Arbeitstag mein eigenes Notebook mit, wodurch ich auf diesem mit dem Rechercheprozess zu Microsoft Dynamics 365 Project Operations beginnen konnte.

Zu besonderen Anlässen - und dieser Tag scheint einer davon gewesen zu sein - setzt sich das anwesende Team zusammen und bekommt von dem Unternehmen ein Mittagessen spendiert. Hierbei konnte ich ersten freundlichen Kontakt mit meinen neuen Kollegen schließen und über deren Aufgabenbereiche erfahren.

\subsubsection*{2. Februar - 9:00 bis 16:45 Uhr}
An meinem zweiten Arbeitstag bekam ich mein fertig eingerichtetes Arbeitsnotebook ausgehändigt, auf dem ich direkt im Anschluss mit einem weiteren neuen Kollegen, Carlos Algarra, virtuell über MS Teams in Kontakt treten durfte. In diesem Meeting wurde mir der Status-Quo der aktuellen Arbeitsverteilung in dem Unternehmen geschildert, sowie die Ergebnisse eines ersten Brainstormings bezüglich der Funktionalitäten des zukünftigen Systems dargelegt.

Nach diesem Meeting habe ich begonnen, auf der Whiteboard-Kollaborationsplattform \href{https://miro.com/de/}{Miro} erste Entitäten in einem Entity-Relation Schema zu entwerfen, auf denen die zukünftigen Prozesse des neuen Systems aufbauen sollen.

\subsubsection*{3. Februar - 9:00 bis 15:10 Uhr}
Zu Beginn des nächsten Tages habe ich die Gestaltung des ER-Schemas fortgesetzt, wobei ich anschließend bei einem erneuten virtuellen Meeting mit Carlos Empfehlungen bezüglich der Verbesserung und des \enquote*{Streamlining} meines bisherigen Ansatzes bekommen habe. Hier wurden auch direkt Fragen beantwortet, die bei der Erstellung des Modells aufgekommen sind. Nachdem das gegebene Feedback implementiert wurde, habe ich final überprüft, ob alle Anforderungen, welche bis dato an den Entwurf gestellt wurden, abgedeckt sind und anschließend eine Dokumentation über die wichtigsten Entitäten des Schemas erstellt, um die Nachvollziehbarkeit des Entwurfs sicherzustellen. 

\subsubsection*{6. Februar - 9:00 bis 15:15 Uhr}
Nachdem der letzte Tag mit der Finalisierung des ER-Schemas geendet hatte, wurde dieser Arbeitstag vollends der Prozessgestaltung gewidmet, wobei darauf geachtet wurde, dass alle modellierten Prozesse auch lückenlos an die Entitäten ansetzen. 

\begin{figure}[h]
    \centering
    \includegraphics[width=0.75\textwidth]{img/miro_erschema.png}
    \caption{Finalisiertes ER-Schema, auf welchem das zukünftige System aufbauen soll.}
\end{figure}

Neben einer oberflächlichen Prozessplanung habe ich an diesem Tag auch begonnen, einzelne Teilprozesse dieser Planung in eine genauere Flowchart-Implementierung umzusetzen.

\subsubsection*{7. Februar - 9:00 bis 15:15 Uhr}
An diesem Arbeitstag habe ich die Teilprozessmodellierung fortgesetzt, wobei der Fokus mehr und mehr in die Detailplanung übergeht. Hieraus entstehen viele neue Fragen, die später mit dem Team abgeklärt werden müssen.

\subsubsection*{8. Februar - 10:00 bis 17:00 Uhr}
Auch an diesem Tag nahm die Detailplanung der Prozesse einen Teil ein. Jedoch bestand die zweite Hälfte des Arbeitstages daraus, ein präsentierbares Organigramm für eine spätere Präsentation des Systems zu erstellen. Dafür habe ich zu Beginn verschiedene Tools (Prezi, Canva, etc.) ausprobiert, da mir die Wahl des zu verwendenden Tools freigestellt wurde. Jedoch landete ich letztendlich bei Microsoft PowerPoint, da diese Applikation alle nötigen Werkzeuge und Vorlagen mitbringt, um optisch ansprechende Organigramme einer Organisation anzufertigen. 

Der Arbeitstag endete mit einem Teams-Call mit Carlos, bei dem ich Feedback zu meinem bestehenden Fortschritt bekommen habe. Dort habe ich erfahren, dass meine Ansätze seiner Meinung nach etwas zu komplex sind, worauf ich mir die Simplifizierung meines Modells als Ziel vorgenommen habe.

\subsubsection*{9. Februar - 8:45 bis 15:15 Uhr}
Der heutige Tag bestand darin, das Organigramm, mit welchem ich gestern begonnen habe, mit Fotos von den Mitarbeitern zu ergänzen, um das Gesamtkonzept ansprechender zu gestalten. Dazu habe ich einen Zugang zu dem unternehmensinternen Fotoarchiv erhalten.
Darauf folgend setzte ich das Feedback, welches ich am vorherigen Tag erhalten habe, in das Modell um.

Im Anschluss hatte ich ein persönliches Meeting mit meinem Betreuer, bei dem ich meine Fortschritte vorstellen und mögliche Fragen stellen konnte. Nach diesem Meeting lud uns mein Betreuer zum Mittagessen ein. Der Nachmittag endete für mich mit meinem ersten Meeting mit allen Mitarbeitern, die an dem Change-Prozess, dessen Kernsystem ich modellieren darf, beteiligt sind. Auch dort habe ich nochmals Feedback zu den Entitäten und Prozessen erhalten und durfte an der Gestaltung von verschiedenen Personas im Unternehmen, auf welche bei der Einführung des neuen Systems besonders geachtet werden muss, teilnehmen.

\subsubsection*{10. Februar - 10:00 bis 16:45 Uhr}
An diesem Tag habe ich mich ausschließlich mit der Implementierung des zuletzt gegebenen Feedbacks beschäftigt, da dies einige grundlegende Umstrukturierungen erforderte.

\subsubsection*{13. Februar - 8:00 bis 15:00 Uhr}
Der heutige Tag begann ebenfalls mit der Fortsetzung der Umsetzung des gegebenen Feedbacks. Im Anschluss recherchierte ich zu Microsoft Dynamics 365 Project Operations, da wir in dem Gruppenmeeting potenzielle Komplikationen, welche zwischen dem gewünschten und dem von Microsoft vorgegebenen System bestehen könnten, besprochen haben.

\subsubsection*{14. Februar - 8:45 bis 15:15 Uhr}
Um weitere mögliche logische Fehler in meinem Prozess- und Entitätenmodell zu finden, habe ich begonnen, den gesamten Prozess mit beispielhaften Daten zu simulieren. Hierbei wurde darauf geachtet, dass möglichst alle Teile der Flow-Diagramme ausgeführt werden, um Fehler in \enquote{seltenen Sondersituationen} zu entlarven.

\begin{figure}[h]
    \centering
    \includegraphics[width=0.75\textwidth]{img/miro_simulation.png}
    \caption{Entities nach der Simulation (CRUD-Operationen) durch verschiedene Subprozesse}
\end{figure}

Nachdem das gesamte Team zusammen zu Mittag gegessen hatte, filmten wir einen kurzen Imageclip, welcher für interne Zwecke bei einem jährlichen \enquote{Kickoff} gezeigt werden soll.

\subsubsection*{15. Februar - 9:15 bis 16:15 Uhr}
An diesem Tag beschäftigte ich mich zu Beginn mit dem Anpassen der Präsentation bzgl. dem Organigramm, da dieses noch etwas zu hierarchisch gewirkt hatte. Daher wählte ich eine holokratische Darstellungsform, um der de-facto flachen Organisationsstruktur gerecht zu werden. 

Im Anschluss nutzte ich meinen Zugang zu der unternehmensinternen Instanz von Microsoft Dynamics 365 Project Operations, um einen rudimentären digitalen Zwilling meiner bisherigen Praktikumstätigkeiten anzulegen. Diesen habe ich dann genutzt, um die Entitäten aus dem modellierten ER-Schema auf reale, bereits bestehende Entitäten in Project Operations umzulegen.

\subsubsection*{16. Februar - 8:45 bis 15:15 Uhr}
Heute beschäftigte ich mich primär erneut damit, das Prozessmodell zu simulieren, da ich bei der letzten Iteration einige Änderungen vorgenommen habe und daher einen \enquote{fehlerfreien Durchlauf} erzielen wollte.

\subsubsection*{17. Februar - 8:45 bis 16:15 Uhr}
Nachdem die Prozesse und Entitäten so weit abgeschlossen sind, begann nun die dokumentierte Fit/Gap-Analyse zwischen dem erstellen Framework und Microsoft Dynamics 365 Project Operations. Diese hat das Ziel, herauszufinden, welche Elemente des selbst gestalteten Prozesses in das Tool übernommen werden können, sowie Segmente, bei denen es potenziell Probleme geben könnte.

Um diese Herausforderung anzugehen, beschäftigte ich mich zu Beginn mit verschiedenen Ansätzen, wie Fit/Gap-Analysen im Business-Kontext normalerweise angegangen werden. Hierbei ist das Wissen, welches ich in der VL Betriebliche Standardsoftware bzgl. Fit/Gap-Analysen gewonnen habe, von Nutzen. Im Anschluss erstellte ich, basierend auf meinen gestalteten Prozessen, eine Liste an beschriebenen Anforderungen, auf welche Project Operations anschließend geprüft wird, um Fits und Gaps auszumachen.

\subsubsection*{20. Februar - 8:45 bis 15:15 Uhr}
An dem heutigen Tag habe ich mich primär damit beschäftigt, Microsoft Dynanmics Project Operations auf das zuletzt erstellte Anforderungsprofil zu prüfen. Im Fokus standen dabei Solutions, welche das Erstellen von benutzerdefinierten Entitäten und Feldern ermöglichen.

\begin{figure}[h]
    \centering
    \includegraphics[width=0.75\textwidth]{img/fit_gap.png}
    \caption{Auszug aus der FIT/GAP-Analyse für Microsoft Project Operations mitsamt Anforderungsprofil}
\end{figure}

\subsubsection*{21. Februar - 8:00 bis 15:15 Uhr}
Nachdem ich mir eine eigene virtuelle Umgebung mitsamt Microsoft Dataverse Datenbank angelegt habe, konnte ich mich Microsoft PowerApps widmen. In diesem Tool ist es möglich, Felder und Entitäten benutzerdefiniert zu bearbeiten, womit ich versucht habe, Dynamics 365 Project Operations auf die Anforderungen des Entity-Modells zu trimmen.

\subsubsection*{22. Februar - 8:45 bis 15:15 Uhr}
An diesem Tag versuchte ich weiter, das bestehende Project Operations an den gestalteten Prozess anzugleichen. Hierbei stoß ich auf einige Architekturprobleme, welche in dem nächsten Meeting mit den verantwortlichen Kollegen besprochen werden müssen.

\subsubsection*{23. Februar - 8:45 bis 15:30 Uhr}
Heute setzte ich mich mit den Möglichkeiten auseinander, Dynamics 365 Dashboards zu erstellen. Diese sollen den Mitarbeitern direkt und ohne Submenüs ihre verbleibende Kapazität und die zugewiesenen Aufgaben darstellen. Der Tag endete mit einem Meeting mit allen Beteiligten, bei welchem wir die aufgetretenen Probleme besprochen und gemeinsam einen neuen Plan für das weitere Vorgehen entwickelt haben.

\subsubsection*{24. Februar - 8:45 bis 15:15 Uhr}
Heute erweiterte ich die erstellten Dashboards und recherchierte bzgl. eines externen Tools, mit welchem ER-Diagramme aus Microsoft Dataverse Datenbanken erstellt werden können. Dies wird in Zukunft nützlich sein, wenn es darum geht, herauszufinden, welche Komponenten von Project Operations für das Projekt nutzbar sind und welche Komponenten evtl. substituiert werden müssen.

\subsubsection*{27. Februar - 8:45 bis 15:15 Uhr}
An diesem Tag setzte ich mich mit PowerBI-Dashboards, der zugehörigen Skriptsprache DAX und dem bestehenden Datenmodell auseinander, um ein optisch-ansprechendes Dashboard zu erzeugen, welches die  Arbeitsstunden, die ein Mitarbeiter diese Woche bereits alloziert hat, anzeigt.

\subsubsection*{28. Februar - 8:45 bis 15:15 Uhr}
Der heutige Tag setzte sich aus dem Erstellen eines Projektmanager-Dashboards, auf dem die Aktivitäten des gesamten Teams dargestellt werden, sowie aus dem Überarbeiten des persönlichen Dashboards zusammen.

\subsection{März}
\subsubsection*{1. März - 9:00 bis 15:45 Uhr}
Nachdem die erstellten Dashboards eine sehr schlechte Performance aufgewiesen hatten, beschäftigte ich mich heute damit, die Berechnungen mit PowerQuery und der zugehörigen relationalen Sprache M umzusetzen, was letztendlich gelungen ist. Das Mitarbeiterdashboard und das PM-Dashboard sind somit fertiggestellt, womit die beiden zugehörigen Anforderungen als FIT markiert werden können.

\subsubsection*{2. März - 8:45 bis 15:15 Uhr}
Heute finalisierte ich die Dashboards und recherchierte bzgl. dem Exportieren meiner bisherigen Arbeit, damit diese dann in der Umgebung, auf welcher Project Operations final laufen soll, importiert werden kann. Nachdem dies erledigt ist, kann ich mich mit der Prozessimplementierung beschäftigen.

\subsubsection*{3. März - 8:45 bis 15:15 Uhr}
An diesem Tag setzte ich mich mit den Möglichkeiten bzgl. der Implementierung der Background-Logik auseinander. Dabei ist einerseits der Zugang mittels PowerAutomate Flows möglich, jedoch auch die WebAPI, welche - auf dem REST-Konzept aufbauend - CRUD-Operationen zulässt. Hierbei läge der Vorteil darin, dass keine Kosten pro Flow, wie in PowerAutomate, anfallen würden.

\subsubsection*{6. März - 8:45 bis 15:45 Uhr}
Der Fokus des heutigen Tages lag auf der Implementierung von Validierungsprozessen in JavaScript. Damit können diese als Teil der Inputformulare in PowerApps ausgeführt werden, was eine Inputvalidierung direkt bei der Eingabe und nicht erst nach der Speicherung möglicherweise fehlerhafter Datensätze ermöglicht. Darüber hinaus habe ich die Dashboards so ergänzt, dass diese auch dann sinnvolle Inhalte anzeigen, wenn diese Woche noch gar keine Buchungen oder Task-Zuweisungen stattgefunden haben.

\subsubsection*{7. März - 8:45 bis 15:15 Uhr}
Heute begann ich mit dem Umsetzen des Status Flags, welcher für jede Projektaufgabe gesetzt ist und welcher überwiegend automatisch gemanaged werden soll. Die automatischen Prüfungen wurden erneut in JavaScript umgesetzt, wobei die erstaunlich gute Performance bei Dataverse Datenbankabfragen positiv auffiel. Somit lassen sich die Möglichkeiten für den Nutzer, Felder auf nicht zulässige Art und Weise zu verändern, minimieren.

\subsubsection*{8. März - 8:45 bis 15:15 Uhr}
Dieser Tag war überwiegend von Recherchen bzgl. der Ausführung von Microsoft Dataverse Actons über die JavaScript API geprägt. Hierbei fiel auf, dass die Dokumentation, welche Microsoft für derartige Anwendungsfälle zur Verfügung stellt, tendenziell eher unzureichend sind. Dies könnte damit zusammenhängen, dass die gleichen Aktionen auch über das kostenpflichtige PowerAutomate mit Dataverse Flows erreicht werden kann.

\subsubsection*{9. März - 8:45 bis 15:15 Uhr}
Das Ziel des heutigen Tages war es, Tasks über die JavaScript API erstellen zu können, sowie diese in eine hierarchische Zuordnung zu bringen. Dieses Thema stellte sich als überaus fordernd heraus, da die Dokumentation seitens Microsoft nicht für die Durchführung mit JavaScript verfügbar ist. Jedoch endete der Tag mit dem erfolgreichen Anlegen von Tasks über JS.

\subsubsection*{10. März - 8:15 bis 14:30 Uhr}
Nach dem gestrigen Erfolg habe ich heute die komplette Logik der automatisch generierten Dependency-Tasks in JavaScript umgesetzt. Nun muss sich der Anwender des Systems nicht mehr selbst darum kümmern, dass notwendige Tasks zur Dokumentation und für das Qualitätsmanagement getaner Arbeit angelegt werden.

\subsubsection*{13. März - 8:45 bis 15:15 Uhr}
Nach dem Wochenende beschäftigte ich mich mit Bugfixes bei der JavaScript Logik der Formulare, aber auch bei den PowerBI-Dashboards, da diese noch nicht vollständig fehlerfrei funktionierten, wenn nach dem Wochenende weder Tasks noch Bookings für einen Benutzer eingetragen waren.

\subsubsection*{14. März - 8:45 bis 14:15 Uhr}
An diesem Tag beschäftigte ich mich primär mit dem Bugfixing bestehender Logiken in JavaScript. Der Fokus lag dabei besonders auf den verschiedenen Situationen, in welchen bestimmte Felder für die Bearbeitung gesperrt sein, oder zur Bearbeitung freigegeben sein sollen.

\subsubsection*{15. März - 8:45 bis 17:00 Uhr}
Der heutige Tag war vormittags von der Implementierung einer Statusanzeige für Buchungen, bei welcher Mitarbeiter bei der Erstellung einer Zeitbuchung ersehen können, wie viel Zeit ihnen noch zur Verfügung steht. Am Nachmittag haben wir bei einem Teammeeting das bestehende System besprochen, wobei Feedback zur Implementierung gegeben und weitere Termine für die Zukunft besprochen wurden.

\subsubsection*{16. März - 8:45 bis 15:15 Uhr} 
Nach gegebenem Feedback machte ich mich nun an die Arbeit, dieses umzusetzen. Dabei wurde mir empfohlen, das, von mir bisher nicht angewandte Konzept der Workflows und Geschäftsregeln anzuwenden, um nicht alle Logiken in JS implementieren zu müssen.

\subsubsection*{17. März - 8:45 bis 15:15 Uhr}
Heute migrierte ich die erstellten Project Operations Customizations in eine neue Testumgebung, nachdem die Laufzeit der alten Umgebung erreicht war. Geplant war, dass das System heute in die Umgebung des Unternehmens migriert wird, was sich jedoch aufgrund nachzuholender Aktualisierungen auf diesem verzögert.

\subsubsection*{20. März - 8:45 bis 16:30 Uhr}
An diesem Tag setzte ich das Feedback bezüglich der Dashboards um. Hierbei handelt es sich um die Anforderung, gebuchte Tasks und Buchungen nicht nur für die aktuelle Woche, sondern auch für nachfolgende Wochen einsehen zu können. Desweiteren betrieb ich Recherche bzgl. Drag'n'Drop Dashboards in PowerBI, welche das Task-Assignment vereinfachen sollten. Dies stellt sich jedoch als unmöglich heraus, weswegen die Lösung für das einfache Task-Assignment wohl ein selbstgeschriebener IFrame, welcher die verfügbaren Tasks anzeigt und die entsprechende Dataverse-Aktion im Anschluss durchführt, sein wird.

\subsubsection*{21. März - 8:45 bis 15:15 Uhr}
Heute finalisierte ich das Task-Assignment über das Task-Formular, womit Mitarbeiter nun Aufgaben, welche ihnen in dem Dashboard vorgeschlagen werden, schnell zuweisen können. Die direkte Zuweisung mittels eines IFrames im Dashboard erscheint eher schwierig, da ein sogenanntes \enquote*{System-Dashboard} notwendig ist, um JavaScript-Aktionen in einem eingebetteten IFrame auszuführen. Die Kombination von PowerBI und IFrames ist in System-Dashboards jedoch kompliziert.

\subsubsection*{22. März - 8:45 bis 15:15 Uhr}
An diesem Tag beschäftigte ich mich erneut mit den IFrames, da ich nach kurzem Testen darauf gestoßen bin, dass doch Möglichkeiten bestehen, auf die Microsoft Xrm-Bibliothek innerhalb eines IFrames auf einem Dashboard zuzugreifen. Entsprechend habe ich ein Dynamics Dashboard mit HTML, CSS und JavaScript \enquote*{nachgebaut}, mit welchem man mittels eines einfachen Klicks einen freien Task sich selbst zuordnen kann.

\subsubsection*{23. März - 8:45 bis 15:15 Uhr} 
Heute finalisierte ich das selbsterstellte HTML-Dashboard, welches mittels eines IFrame in Dynamics eingebettet werden kann. Der Fokus lag dabei darauf, die User Experience der regulären Dynamics Anwendungen möglichst zu imitieren, was relativ gut gelungen ist. Im Anschluss beschäftigte ich mich mit der Status-Update-Logik für Tasks in JavaScript, da diese noch einige Ungereimtheiten aufweist. Sobald diese Aufgabe abgeschlossen ist, werde ich in die Testphase übergehen, um weitere Bugs zu finden.

\subsubsection*{24. März - 9:00 bis 16:00 Uhr}
Nach letzten Optimierungen bei dem PowerBI Dashboard für Mitarbeiter, den Task-Auswahl-, sowie Übersichts-Dashboards und bei der Statusfeld-Logik, habe ich heute das Projektmanager-Dashboard neu modelliert. Nun ist es auch hier möglich, die Auslastung einzelner Mitarbeiter, sowie der gesamten Belegschaft über mehrere Wochen hinweg einzusehen, um einen besseren Überblick über verbleibende Kapazitäten in nächster Zeit zu haben.

\subsubsection*{27. März - Ausfall wegen Krankheit}

\subsubsection*{28. März - 9:00 bis 16:00 Uhr}
Der heutige Tag setzte sich aus dem Optimieren beider Dashboards, sowie dem Testen des Gesamtsystems auf Fehler, zusammen. Bei den Dashboards gab es noch ein Problem mit Zeitbuchungen, die noch nicht genehmigt waren. Diese wurden bereits angezeigt, aber noch nicht von dem verbleibenden Aufwand abgezogen, weshalb letztendlich zu viel Zeit in den Dashboards angetragen wurde. Der gleiche Fehler bestand bei dem Erstellungsformular für Zeitbuchungen. 

\subsubsection*{29. März - 9:30 bis 16:30 Uhr}
Heute aktualisierte ich weitere Teile der JavaScript Logik bezüglich der Status-Aktualisierungen. Hierbei bestand die Herausforderung darin, dass QM- und Dokumentationstasks nun in ihrer Dependency-Kette erkannt werden müssen, um festzustellen, ob eine Buchung in einen QM-Task den \enquote*{Elterntask} auf den Status \enquote*{QM} setzen soll. Im Anschluss hatten Stefan, Carlos und ich ein Meeting, bei dem ich noch einiges an Feedback und Ideen bekommen habe, besonders im Kontext der neu erstellten Funktionalitäten.

\subsubsection*{30. März - 9:15 bis 16:15 Uhr}
An diesem Tag beschäftigte ich mich primär mit dem Feedback, welches ich bei dem gestrigen Meeting bekommen habe. Ich begann damit, das Task-Auswahldashboard in Bezug auf Design und tabellarischer Sortierfunktion zu überarbeiten und einen Fehler bei der Status-Logik zu beseitigen.  

\subsubsection*{31. März - 8:45 bis 13:45 Uhr}
Heute erstellte ich zwei neue Ansichten des PowerBI Dashboards für Projektmanager: Eines, bei welchem die Mitarbeiterauslastung für eine frei-wählbare Woche angezeigt werden kann und eines, bei welchem für jedes Projekt eingesehen werden kann, wie viele der bereits gebuchten Arbeitsstunden dem Kunden verrechnet werden können und wie viele Stunden das Unternehmen selbst tragen muss. Darüber hinaus habe ich noch einen Fehler in dem Booking-Schnellerfassungsformular behoben, nachdem dieses die Warnungs- und Gesamtarbeitsstundenlimits nicht korrekt angezeigt hatte.  

\subsection{April}
\subsubsection*{3. April - 9:00 bis 15:15 Uhr}
Nachdem der letzte volle Monat meines Praktikums begonnen hat, befasste ich mich heute damit, meinen bisher angelegten Quellcode ausführlich zu kommentieren, damit auch Mitarbeiter, die sich nach mir mit dem Projekt beschäftigen, ein gutes Verständnis meines Codes erlangen können. Darüber hinaus finalisierte ich das PowerBI Verrechnungsdashboard, mit welchem ich am 31. März nicht mehr ganz fertig geworden bin.

\subsubsection*{4. April - 8:45 bis 16:15 Uhr}
Der heutige Tag begann mit einer Schulung zu Microsoft Dynamics 365. Im Anschluss habe ich Hilfestellung von einer Kollegin bei dem Einrichten eines Microsoft Dyanmics Workflows erhalten, bei welchem ich auf die Guid einer Tabelle zugreifen muss, was Workflows jedoch nicht standardmäßig unterstützen. Danach begann ich mit der Dokumentation des gesamten Systems, wobei die Dokumentation aus einem Benutzer-, Administrator- und Entwicklerhandbuch bestehen wird.

\subsubsection*{5. April - 9:15 bis 15:15 Uhr}
Heute habe ich mich neben kleineren Fixes bei den Dashboards primär mit der Dokumentation der bestehenden Abläufe beschäftigt, wobei das Kapitel des Benutzerhandbuchs so weit fertiggestellt werden konnte. Deswegen begann ich mit der Beschreibung der einzelnen Komponenten der Dynamics Solution, welche morgen fortgesetzt wird.

\subsubsection*{6. April - 8:45 bis 15:15 Uhr}
An dem heutigen Tag habe ich alle JavaScript Kommentare auf den JSDoc Standard ausgebaut, um ein standardisiertes, leicht-leserliches und in HTML-kompilierbares Kommentar- und Dokumentationsformat für alle Funktionen zu benutzen. Darüber hinaus lag der Fokus erneut auf dem Ausarbeiten der technischen Dokumentation des Gesamtsystems.

\subsubsection*{11. April - 8:45 bis 16:15 Uhr}
Heute habe ich die Dokumentation für die technische Weiterentwicklung des Systems abgeschlossen, das Task-Assignment Dashboard überarbeitet, damit Fehlzuweisungen von Anfang an verhindert werden, Kontakt mit Michael und Stefan bzgl. der Microsoft Project Operations Lizenzen gehabt und meine Solution final bereinigt, bevor diese morgen in UniProCE, also die echte Unternehmensumgebung, übertragen wird.

\subsubsection*{12. April - 8:45 bis 15:15 Uhr}
Der heutige Tag begann mit dem Ändern aller Entity-Prefixes von einem, zu Beginn meiner Customizations automatisch gewählten, Präfix zu dem UniProCE Präfix. Anschließend wurde meine Solution aus meiner Entwicklungsumgebung, welche ich die letzten Monate für das Testen der Funktionalitäten benutzte, exportiert und in die UniProCE Umgebung importiert. Im Anschluss wurden kleinere Fehler, welche in der Testumgebung nie aufgefallen sind, deutlich, welche nun behoben werden müssen. Ist dieser Prozess abgeschlossen, kann der Testprozess mit realen Mitarbeitern und realen Projekten beginnen.

\subsubsection*{13. April - 8:45 bis 15:15 Uhr}
An diesem Tag musste ich meine Solution wieder aus dem System exportieren und anschließend löschen, da es noch Probleme mit Project Operations gegeben hatte, bei welchem erst die eigene Solution-Struktur angepasst werden musste. Dies habe ich genutzt, um noch einige Anpassungen an den Metadaten meiner Solution vorzunehmen, um diese dann morgen wieder importieren zu können.

\subsubsection*{14. April - 8:00 bis 14:00 Uhr}
An dem heutigen Tag setzte ich das PowerBI Datenmodell größtenteils auf UniProCE um und versuchte in der Früh, Bugs in Project Operations, welche ebenfalls durch UniProCE verursacht werden, zu beheben.

\subsubsection*{17. April - 7:15 bis 15:15 Uhr}
Heute habe ich das PowerBI Projektmanager Dashboard auf die Datenbasis der Umgebung UniProCE umgesattelt, sowie die Frage, welche PowerBI Lizenz welche Angestellten benötigen, geklärt. Im Anschluss habe ich ein Gespräch mit Carlos zu dem aktuellen StatusQuo des Projekts geführt und den Code für die automatische Erstellung von Projekten bei dem Anlegen von Sales-Orders erstellt.

\subsubsection*{18. April - 7:15 bis 15:15 Uhr}
An diesem Tag habe ich den Code für die automatische Task-, bzw. Projekt-Erstellung bei dem Anlegen neuer Cases erstellt. Dies verlangte eine Änderung in dem Datenmodell, wonach das Skill-Feld, welches jeder Task aufweist, nicht mehr verpflichtend sein kann. Dies liegt daran, dass für die Erstellung eines neuen Tasks sonst zwingend ein Skill verlangt wird, jedoch kein Skill bei der Erstellung eines Case festgelegt wird.

\subsubsection*{19. April - 7:15 bis 15:15 Uhr}
Heute bekam ich in einem erneuten Anruf mit Carlos Input zu Rollup-Feldern, die für Tasks die Summe der Bookings in Stunden, welche an den Task gekoppelt sind, aber auch die der Bookings möglicher untergeordneter Tasks. Darüber hinaus habe ich einen \enquote*{Team besetzen} Button für Projekte implementiert, bei dem alle Teammitglieder mit ihren Wochenstunden und Zielauslastung automatisch als Projektteammitglieder initialisiert werden. Abschließend habe ich begonnen, die Query-Steps in dem Power BI Datenmodell umzubenennen, um die Verständlichkeit zu erhöhen.

\subsubsection*{20. April - 7:15 bis 15:15 Uhr}
An diesem Tag vollendete ich die Umbenennung der Queries im Power BI Datenmodell, verlinkte Cases mit SalesOrders in meiner Solution, testete und debuggte das automatische Erstellen von Projekten und Tasks bei dem Abspeichern eines neuen Cases, fügte der Task-Entität ein weiteres Feld bzgl. der initialen Aufwandsschätzung hinzu und ergänzte das Feld um eine Geschäftsregel, nach der es nur einmal editierbar ist und dann automatisch auf read-only gesetzt wird. Anschließend aktualisierte ich die Dokumentation, sodass ich zwei weitere Kollegen, im Rahmen eines Einführungsgespräches in die technische Umsetzung des Projekts bis dato einweisen konnte, damit dieses nach meinem Berufspraktikum nahtlos fortgesetzt werden kann. 

\subsubsection*{21. April - 7:15 bis 15:15 Uhr}
Der heutige Tag war von Power BI Customizations geprägt. Dabei war das Ziel, das PM-Dashboard um Filtermöglichkeiten bzgl. Namen einzelner Mitarbeiter und Skill der Belegschaft zu erweitern. Während Filtern nach Namen kein Problem darstellt, ist PowerBI nicht in der Lage, nach Elementen einer Liste, welche in einer Tabellenspalte gespeichert ist, zu filtern, woraus sich eine größere Herausforderung ergeben wird.

\subsubsection*{24. April - 8:15 bis 14:15 Uhr}
An diesem Tag habe ich das Problem um die Rollup-Felder gelöst, indem ich auf eine versteckte Hierarchiefunktion gestoßen bin. Darüber hinaus habe ich zusätzliche Task-Felder implementiert, welche wir bei dem letzten Gespräch mit Carlos definiert hatten. Im Anschluss habe ich die Dokumentation aktualisiert und begonnen, die gesamte Solution an den audius Styleguide für Customizations anzupassen.

\subsubsection*{25. April - 8:45 bis 15:15 Uhr}
Heute importierte ich die, nun an den Styleguide angepasste, Solution meines Projekts in das System. Davor musste ich die alte Solution mitsamt der Komponenten einzeln entfernen, da die Solution als \enquote*{unmanaged} installiert wurde, um noch nachträglich Anpassungen vornehmen zu können. Damit auch das Power BI Datenmodell mit den abgewandelten Entitäten funktioniert, habe ich im Anschluss auch dort alle relevanten Query-Steps angepasst. Um diese Änderung auch in der Dokumentation des Systems widerzuspiegeln, wurde auch diese anschließend angepasst.

\subsubsection*{26. April - Freigenommen wegen Verleihung des WKS Leistungsstipendiums}

\subsubsection*{27. April - 8:45 bis 14:15 Uhr}
An diesem Tag verlinkte ich Projektaufgaben und SalesOrder Details, eine Entität aus dem bestehenden CRM-System. Diese Verlinkung ist eine notwendige Parallelstruktur, da die Vorgabe besteht, dass sich an dem existierenden Invoicing-Prozess nichts ändern soll. Ein JavaScript Code erstellt ein CRM Booking automatisch dann, wenn ein neuer Zeiteintrag angelegt wird. Da das Setzen der Eigenschaften \enquote*{verrechenbar} und \enquote*{verrechnet} nun nicht mehr über Zeiteinträge, sondern über die CRM-Bookings stattfinden soll, kümmert sich ein Workflow um das Synchronisieren der funktionsgleichen Felder zwischen beiden Entitäten.

\subsubsection*{28. April - 8:45 bis 14:15 Uhr}
Heute erstellte ich den Link zu zugehörigen SalesOrders bei den automatisch erstellten CRM Bookings. Darüber hinaus fügte ich eine Möglichkeit hinzu, in dem PowerBI Projektmanager*innendashboard nach Fähigkeiten zu sortieren, damit beispielsweise eingesehen werden kann, wie ausgelastet die Entwickler*innen gerade sind. Im Anschluss ergänzte ich die Dokumentation entsprechend und begann, die intern im JS-Code, den Ansichten und dem BI-Datenmodell verwendete \enquote*{\lstinline|msdyn\_effortremaining|} auf die selbst berechnete \enquote*{\lstinline|ud\_remainingcapacity|}.

\subsection{Mai}

\subsubsection*{2. Mai - 8:45 bis 14:15 Uhr}
Am heutigen Tag stellte ich die Änderungen an dem PowerBI-Datenmodell fertig und aktualisierte die Dokumentation. Im Anschluss erstellte ich ein ausführliches ToDo-Dokument, welches die nächsten Aktivitäten, die für die weitere Implementierung des Systems notwendig sind, erläutert. Danach musste ich gespiegelte Rollup-Felder für die Projekt-Entität erstellen, welche die bebuchten, verrechenbaren und verrechneten Stunden pro Projekt zusammenfassen, um auch hier einen Überblick basierend auf den selbst-berechneten Feldern zu bekommen.

\subsubsection*{3. Mai - 8:45 bis 15:45 Uhr}
Der heutige und letzte Tag des Berufspraktikums bestand aus einem Anruf mit Carlos, Mateja und Roland, in dem wir die ToDos für die nächste Zeit zusammen abgesteckt haben. Darüber hinaus habe ich alle Daten, darunter die Fit/Gap-Analyse, die erstellte Präsentation, die zwei PowerBI Reports, die JavaScript- und HTML-Webressourcen, sowie die Dokumentation und das ToDo-Dokument in das Filesharing System hochgeladen, damit das Projekt bestmöglich fortgesetzt werden kann. Hierzu habe ich die Solution in ihrer letzten Form exportiert. Am Nachmittag wurde das neue System allen Mitarbeiter*innen vorgestellt, womit mein Berufspraktikum endet.
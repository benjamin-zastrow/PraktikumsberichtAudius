\ihead{Vorstellung der Projektaufgabe}
\section{Vorstellung der Projektaufgabe}
\label{sec:VorstAufgabe}

Die Projektaufgabe, welche ich im Rahmen dieses Praktikums lösen soll, beschäftigt sich mit der effizienteren Allokation von Mitarbeiter*innenressourcen, welche aufgrund wachsender Belegschaftszahlen allmählich nicht mehr von der Management-Ebene betreut werden kann. Das Ziel liegt also darin, ein, mit der bestehenden Microsoft Dynamics 365 Umgebung kompatibles, System zu entwerfen, zu planen und umzusetzen, welches in der Lage sein soll, eingehende Aufgaben benötigten Fähigkeiten und anschließend verfügbaren Mitarbeiter*innen, welche über diese Fertigkeiten verfügen, zuzuordnen.

Die Mittel, die dabei eingesetzt werden, belaufen sich auf die vereinbarten 360 Stunden Arbeitszeit, ein Notebook des Unternehmens, Zugang zu der Planungssoftware Miro, sowie Zugang zu dem bestehenden System für Entwicklungszwecke.

Bezüglich des zeitlichen Ablaufs des Projekts wird die vorgegebene Anforderungsliste zunächst in ein ER-Datenmodell umgewandelt, an welches dann konzeptionell high- und low-level Prozesse angeknüpft werden. Im Anschluss wird das Datenmodell und die zugehörigen Prozesse in einer Test- und Entwicklungsumgebung implementiert, um anschließend getestet zu werden. Dabei ist anzunehmen, dass weitere Anforderungen im Laufe der Umsetzung hinzukommen werden, wodurch der Prozess insgesamt einer iterativen Arbeitsweise folgen wird.
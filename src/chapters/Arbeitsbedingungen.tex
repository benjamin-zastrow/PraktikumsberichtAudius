\ihead{Erfahrungen und Arbeitsbedingungen}

\section{Erfahrungen und Arbeitsbedingungen}
\label{sec:arbeitsbedingungen}

Das Arbeiten in der audius GmbH wird generell in einem freiwillig-hybriden Modus durchgeführt. Dabei ist es allen Mitarbeitenden, die dies möchten, freigestellt, ihren Aufgaben entweder daheim oder an dem Standort in Freilassing nachzukommen. Das Unternehmen verfügt dabei über ein digitales Zeiterfassungssystem, mit welchem Arbeitsstunden, sowie getätigte Aktivitäten dokumentiert werden müssen. Den Angestellten im Bereich Consulting und Software-Entwicklung wird dabei ein Firmennotebook zur Verfügung gestellt, mit dem man sich, unabhängig von dem eigenen Aufenthaltsort, in alle Firmensysteme einloggen kann.

Allgemein ist das Arbeitsklima sehr offen und kollegial gestaltet. Die Geschäftsführung gibt sich Mühe, einen engen Kontakt zu den Mitarbeitenden zu halten. Bei Fragen können spontan bilaterale Meetings ausgemacht werden und das gesamte Team arbeitet darauf hin, sich gegenseitig bei Problemen zur Seite zu stehen. Es ist üblich, sich bei Problemen direkt an die entsprechende Person zu wenden, womit die Kommunikationswege sehr kurz ausfallen.

Die Arbeitszeiten sind - dem hybriden Setup geschuldet - überwiegend frei selbst festzulegen, solange gewährleistet wird, dass die vertraglich festgelegte Wochenstundenzahl langfristig eingehalten wird. 

Da sich das Team der audius GmbH nicht immer in Freilassing befindet, wird ein signifikanter Anteil der teaminternen Kommunikation über Onlinetelefonate abgewickelt. Nachdem das Team zusätzlich einige internationale Mitglieder hat, werden diese Gespräche somit auch häufig auf Englisch abgehalten, was jedoch mit einem Grad an Selbstverständnis betrachtet wird.

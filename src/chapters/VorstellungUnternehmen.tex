\ihead{Vorstellung des Unternehmens}
\section{Vorstellung des Unternehmens}
\label{sec:vorstellungunternehmen}

Die audius SE wurde 1991 gegründet und ist bis heute im Bereich Informationstechnologie, Software, Security, Managed Services, Consulting, Service Center Solutions, Professional Services und Mobilfunk tätig. Das Unternehmen ist dabei auf Business-Solutions spezialisiert. Die audius SE an sich ist dabei eine Holdinggesellschaft, welche zwölf Tochterunternehmen an über 20 Standorten mit ca. 500 Mitarbeitern weltweit besitzt. Die audius GmbH in Freilassing ist eine dieser Tochtergesellschaften.

\begin{figure}[h]
    \centering
    \includegraphics[width=0.25\textwidth]{img/audius_logo.png}
    \caption{Logo des Unternehmens}
\end{figure}

Der Standort Freilassing wurde 1974 unter dem Namen Unidienst GmbH gegründet und gehört erst seit 2020 zur audius Gruppe. Das Team in Freilassing besteht zurzeit aus knapp unter 20 Mitarbeitern, wobei die Angestelltenzahl stetig wächst. Geschäftsführer ist Bert Enzinger, der die Unidienst GmbH zugleich auch begründet hat. Jedoch übernimmt Dipl.-Ing. Stefan Wambacher die operative Geschäftsführung, wodurch das Unternehmen einer Doppelspitze untersteht.

Das Unternehmen bedient überwiegend Kunden im KMU-Segment und beliefert diese mit maßgeschneiderten, selbstentwickelten Softwarelösungen für die Branchen der Produktion, Handel und Finanzdienstleistungen. Darüber hinaus stehen Projektberater zur Verfügung, an die sich Kunden wenden können, um auf dem Weg von der Idee bis zur fertigen Implementierung einer Unternehmens- oder Branchenlösung begleitet zu werden. Der Betrieb setzt komplett auf den Microsoft Technology Stack, also auf Dynamics 365, Power BI, sowie die Power Platform. Das Leistungsspektrum der audius GmbH in Freilassing setzt sich dabei aus betrieblicher Prozessanalyse, Projektmanagement nach dem SCRUM-Modell, Systemanpassung, Training, Customer Service und Datenmigration zusammen.
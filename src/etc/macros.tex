% =========================================================
%  Definition of macros
%
%  You won't need most of them\ldots
%  You can remove this file or define your own macros in here.
% =========================================================

% Abkürzung mathbf für einzelne Ziffern - muss innerhalb $$ oder \math sein
% Bsp: 
%      Die Zahlen unterscheiden sich in der Zehnerstelle: $ 5\*33 \neq 5\*43 $
\def\*#1{\mathbf{#1}}

\newcommand{\todo}[1]{\marginpar{\tiny\textcolor{red}{#1}}}

% Turn/rotate font by 90°. Needs package {}graphicx} or {rotating}
% e.g. \side{Sideway text}
\newcommand{\side}[1]{\rotatebox{90}{#1}}
\newcommand{\sh}[1]{\side{\textbf{#1}}} % Sideways bold

% Needs package {pdfpages}
% Includes whole pdf, generating title for toc and label
%  arg1: Section-title for main document
%  arg2: file path of included document without file ending (.pdf)
\newcommand{\addAppendix}[2]{
    \ihead{\appendixname -- #1} %\appendixname = Anhang / Appendix, depending on language used
    \includepdf[scale=0.9,trim=0cm 2cm 0cm 1cm,pages={1},pagecommand={\section{#1}\label{#1}}]{#2}
    \includepdf[scale=0.9,trim=0cm 2cm 0cm 1cm,pages=2-,pagecommand={}]{#2}
}


% Für Frageboegen mit freier Antwortmoeglichkeit:
% https://tex.stackexchange.com/questions/11305/questionnaire-template
% Use: \freequestion Das ist die Frage? \par
\def\freequestion#1\par{#1\par\nobreak
    \begingroup\nobreak
    \advance\leftskip by 2pc
    \hrule width 0pt height 1.7\baselineskip\hrulefill
    \hrule width 0pt height 1.7\baselineskip\hrulefill
    \par
    \medskip
    \endgroup
}

% Für Frageboegen mit 5 Antwortmoeglichkeiten. In plain LaTeX ohne package:
% https://gist.github.com/lindig/3935127350488c42b8cd40b317e4cbec
% Use: \question Das ist die Frage \par \agree
\def\question#1\par#2\par{\hbox to \hsize
    {\vbox{\hsize=0.72\hsize #1\dotfill}\quad#2\hfil}\medskip\goodbreak}
\def\boxit#1{\hbox{\lower0.7ex\vbox{\hrule\hbox{\vrule\kern1pt
                \vbox{\kern1pt\hbox to 1.3em
                    {\small\strut\hfil #1\hfil}\kern1pt}\kern1pt\vrule}\hrule}}}
\newdimen\scalewidth
\scalewidth=0.25\hsize 
\def\fiveboxes#1#2#3#4#5{\hbox to\scalewidth
    {\boxit{#1}\hfil\boxit{#2}\hfil\boxit{#3}\hfil%
        \boxit{#4}\hfil\boxit{#5}}}
\def\boxes{\fiveboxes{}{}{}{}{}\ignorespaces}%
\def\boxOne{\fiveboxes{X}{}{}{}{}\ignorespaces}%
\def\boxTwo{\fiveboxes{}{X}{}{}{}\ignorespaces}%
\def\boxThree{\fiveboxes{}{}{X}{}{}\ignorespaces}%
\def\boxFour{\fiveboxes{}{}{}{X}{}\ignorespaces}%
\def\boxFive{\fiveboxes{}{}{}{}{X}\ignorespaces}%
\def\scale#1#2{%
    \setbox0=\hbox{\boxes}%
    \setbox1=\hbox to \wd0{\small\strut\hfill #2 $\to$}%
    \setbox2=\hbox to \wd0{\small\strut $\gets$ #1 \hfill}%
    \vbox{\medskip\box1\box2\kern2pt\box0}}
\def\agree{\scale{stimme gar nicht zu}{stimme völlig zu}}
\def\myscale#1#2{%
    \setbox1=\hbox to \wd0{\small\strut\hfill #2 $\to$}%
    \setbox2=\hbox to \wd0{\small\strut $\gets$ #1 \hfill}%
    \vbox{\medskip\box1\box2\kern2pt\box0}}
%%%%%


% \blackout{arg1} um Text zu schwärzen
\newlength{\blackoutwidth}
\newcommand{\blackout}[1]
{%necessary comment
    \settowidth{\blackoutwidth}{#1}%
    \rule[-0.3em]{\blackoutwidth}{1.125em}%
}


% current Month / Year for titlepage
\newcommand{\dateDE}{\ifcase \month \or
    Januar\or%
    Februar\or%
    März\or %
    April\or%
    Mai \or%
    Juni\or%
    Juli\or%
    August\or%
    September\or%
    Oktober\or%
    November\or%
    Dezember%
    \fi~\number\year}

\newcommand{\dateEN}{\ifcase \month \or
    January\or%
    February\or%
    March\or %
    April\or%
    May \or%
    June\or%
    July\or%
    August\or%
    September\or%
    October\or%
    November\or%
    December%
    \fi~\number\year}
